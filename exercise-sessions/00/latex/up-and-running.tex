\begin{frame}[fragile]

\frametitle{Get An Emerald Node Up and Running}

\begin{itemize}

\footnotesize

\item The Docker image prepared for this course is on Docker
Hub:\\[0.5em] \url{https://hub.docker.com/r/portoleks/in5570v20/}

\item Once you have Docker installed, you can run it as follows:

\vspace{-0.2in}

\begin{lstlisting}[language=bash]
$ docker run -it --rm portoleks/in5570v20:latest
\end{lstlisting}

\item \textbf{NB!} Windows users see also next slide

\item This lands you in a BASH shell, having the following binaries:

\begin{itemize}

\footnotesize

\item \texttt{ec} (Emerald compiler)

\item \texttt{emx} (Emerald virual machine / execution engine)

\end{itemize}

\item \textbf{NB!} The file-system is ephemeral; once you exit (type
\texttt{exit}, or press Ctrl+D), the files you create here are lost!

\item To run with \textbf{\underline{your working
directory}}\footnote{Make sure you are in a project directory before
you run this command} mounted, do this instead:

\vspace{-0.2in}

\begin{lstlisting}[language=bash]
$ docker run -it --rm \
    --volume "$(pwd):/home/docker/src/" \
    --workdir "/home/docker/src/" \
    portoleks/in5570v20:latest
\end{lstlisting}

\item For more on what this does see the README at

\url{https://hub.docker.com/r/portoleks/in5570v20/}

\end{itemize}

\end{frame}


\begin{frame}[fragile]

\frametitle{Windows Users}

\begin{itemize}

\item If you are using Windows 10 Home Edition (or similar):

\begin{itemize}

\item Download and install Docker Toolbox:

\url{https://docs.docker.com/toolbox/overview/}

\item Make sure your folder with Emerald code (from where you also
execute the \texttt{docker run} command), is under
\texttt{C:\textbackslash{}Users}

\item Make sure that folder is not marked read-only (placing it under
your user directory should work)

\end{itemize}

\item If you are using Git BASH, you should be OK with the above command.

\item If you are using PowerShell, use this command instead:

\vspace{-0.2in}

\begin{lstlisting}[language=bash]
$ docker run -it --rm `
    --volume "${PWD}:/home/docker/src/" `
    --workdir "/home/docker/src/" `
    portoleks/in5570v20:latest
\end{lstlisting}

\end{itemize}

\end{frame}

\begin{frame}[fragile]

\frametitle{Print \texttt{Hello, World!}}

Here is a program with some observable behaviour:

\inputsrc{00-hello-world.m}

To compile and run:

\begin{lstlisting}[language=bash]
$ ec hello.m    # Assuming you call the above file hello.m
$ emx hello.x   # Assuming ec went well, you'll get a hello.x
\end{lstlisting}

\end{frame}


\begin{frame}[fragile]

\frametitle{Print \texttt{Hello, World!} Everywhere (1/2)}

This code asks every node to print \texttt{Hello, World!}\\ in each
their own standard output stream:

\inputsrc{01-hello-all.m}

\end{frame}

\begin{frame}[fragile]

\frametitle{Print \texttt{Hello, World!} Everywhere (2/2)}

To compile and run (on 3 nodes):

\begin{lstlisting}[language=bash]
$ ip addr                   # Determine the IP address of master
...
... eth0...
  inet 172.17.0.2...        # Here it is, under eth0, inet
$ emx -R                    # Start an Emerald master node
Emerald listening on port 17099...
\end{lstlisting}

\begin{lstlisting}[language=bash]
# Start another node
$ emx -R172.17.0.2:17099    # The port is (often) optional
\end{lstlisting}

\begin{lstlisting}[language=bash]
# Start another node, and run helloall.x
$ ec helloall.m             
$ emx -R172.17.0.2:17099 helloall.x
\end{lstlisting}

\textbf{NB!} Don't use a space between \texttt{-R} and the node
identifier\\ (e.g., write \texttt{-Rlocalhost},
\textbf{\underline{not}} \texttt{-R localhost}).

\end{frame}
