\setbeamerfont{title}{size=\LARGE}

% Logo
%\addtobeamertemplate{title page}{%
%  \begin{textblock}{20}(11.5,-0.5)%
%    \includegraphics[height=1in]{figures/white-logo}%
%  \end{textblock}%
%}

\usepackage[T1]{fontenc}
%\usepackage[sfdefault]{AlegreyaSans} %% Option 'black' gives heavier bold face
%% The 'sfdefault' option to make the base font sans serif
%\renewcommand*\oldstylenums[1]{{\AlegreyaSansOsF #1}}

\usepackage[scaled=0.85]{beramono}
\setbeamertemplate{navigation symbols}{} % strip navigation symbols
%\usefonttheme{serif} % use the sans-serif fonts (Palatino)
\setbeamerfont{title}{series=\bfseries,parent=structure}
\setbeamerfont{frametitle}{series=\bfseries,parent=structure}

\usepackage{textpos}
%\usepackage{paralist}
%\usepackage{enumitem}
\usepackage{expdlist}
\usepackage{subfigure}
\usepackage{amssymb, amsmath, amsthm}
\everymath{\displaystyle}

\usepackage[T1]{fontenc}
\usepackage{mathpazo}

\definecolor{mgcolor}{RGB}{230,230,230}
\definecolor{bgcolor}{RGB}{255,255,255}
\definecolor{fgcolor}{RGB}{0,0,0}

\newcommand{\code}[1]{\texttt{#1}}
\newcommand{\mono}[1]{\texttt{#1}}
\newcommand{\textt}[1]{\ensuremath{\text{\mono{#1}}}}
\newcommand{\mathmono}[1]{\ensuremath{\text{\mono{#1}}}}
\newcommand{\nonterm}[1]{\ensuremath{\text{\mono{<#1>}}}}
\newcommand{\term}[1]{\ensuremath{\text{\mono{`#1'}}}}
\newcommand{\any}[0]{\ensuremath{\left\langle\bigtriangleup\right\rangle}}
\newcommand{\D}{$\Delta$}

\newcommand{\makeTable}[6][htbp!]
{
	\begin{table}[#1]
	\centering
	\begin{tabular}{#4}
	#5\\
	#6\\
	\end{tabular}
	\end{table}
}

\newcommand{\includeFigure}[4][htbp!]
{
	\begin{figure}[#4]
	\centering
	\IfDecimal{#1}
	{
		\includegraphics[scale=#1]{figures/#2}
	}
	{
		\includegraphics[#1]{figures/#2}
	}
	\caption[]{#3}
	\label{figure:#2}
	\end{figure}
}

\usepackage{clrscode3e}
\usepackage{verbatim}
\usepackage{listings}
\lstset
{
  language=bash,
	tabsize=2,
	numbers=none,
	breaklines=true,
  backgroundcolor=\color{mgcolor},
%	foregroundcolor=\color{bgcolor},
	framexleftmargin=0.05in,
	basicstyle=\ttfamily\footnotesize,
	numberstyle=\tiny,
%  keywordstyle=\color{green},
%	stringstyle=\color{red},
%	commentstyle=\color{ForestGreen},
  mathescape=true,
  captionpos=b,
  columns=fullflexible,
  breakatwhitespace,
  extendedchars=true,
%  escapeinside={(*@}{@*)},
  mathescape=false,
  keepspaces,
  emphstyle={\bf},
  showstringspaces=false
}

\setbeamercolor{title}{fg=fgcolor}
\setbeamercolor{frametitle}{fg=fgcolor}
\setbeamercolor{normal text}{fg=fgcolor}
\setbeamercolor{background canvas}{bg=bgcolor}
\setbeamercolor{normal text}{fg=fgcolor}
\setbeamercolor{itemize item}{fg=fgcolor}
\setbeamercolor{itemize subitem}{fg=fgcolor}
\setbeamercolor{description item}{fg=fgcolor}
\setbeamercolor{enumerate item}{fg=fgcolor}
\setbeamercolor{block title}{fg=fgcolor}

\setbeamercolor{bibliography item}{fg=fgcolor}
\setbeamercolor{bibliography entry author}{fg=fgcolor}
%\setbeamertemplate{bibliography item}[text]

\usepackage{fancyvrb}

\newcommand{\question}[1]%
{%
  \begin{center}%
  \Large \textbf{#1}%
  \end{center}%
}

\newcommand{\answer}[1]%
{%
  \begin{flushright}%
  --- #1%
  \end{flushright}%
}

\usepackage{xparse}

\NewDocumentCommand\mathWrap{mmm}{\ensuremath{%
  \mathopen{}\left#1#2\right#3\mathclose{}%
}}

\NewDocumentCommand\set{m}{%
  \mathWrap{\{\mathrel{}}{#1\mathrel{}}{\}}%
}

\NewDocumentCommand\floor{m}{%
  \mathWrap{\lfloor\mathrel{}}{#1\mathrel{}}{\rfloor}%
}

\NewDocumentCommand\card{m}{%
  \mathWrap{|\mathrel{}}{#1\mathrel{}}{|}%
}

\NewDocumentCommand\p{m}{%
  \mathWrap{(}{#1}{)}%
}

\usepackage{bashful}
